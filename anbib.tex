\documentclass{article}
\usepackage{filecontents}
\usepackage[authoryear]{natbib}
\usepackage{usebib}

\newcommand{\printarticle}[1]{\citeauthor{#1}, ``\usebibentry{#1}{title}''}
\begin{document}
\title{GEOG701 Annotated Bibliography}
\author{Dylan Skrah}

\maketitle

\section{Week 2}
\subsection{Theme: Modeling School Segregation}
\begin{itemize}

\item \cite{monarrez2021RaciallyUnequal}
\subitem
\item This paper evaluates ethnic and racial compositions of neighboring schools to find discontinuities. The question is to what extent do neighboring schools segregate their populations and school resources (such as staffing). The authors find over 2000 pairs of neighborhoring schools that are vastly different from each other in ethnic composition and resources, suggesting that school attendance boundaries (SABs) are drawn in ways that amplify segregation, intentionally or otherwise. They link existing inequalities to those created by the New Deal's Home Owners' Loan Corporation redlining policies using historical maps. While they use SABs extensively, they source their data privately, rather than using the public School Attendance Boundary Survey.

\item \cite{saporito2016IrregularlyShaped}
\item This paper contextualizes the 'school attendance zone' or 'school attendance boundary' (SAB) as a concept in education research and demography. The argues establish the benefits of diverse schools and argue that the drawing of SABS is one of the best tools available to administrators to increase diversity. They operationalize irregularity (as in shapes) and show that the more irregularly drawn (i.e. not rectangular) shapes contribute to school diversity. Compact, rectangular SABs tend to mirror the demographic makeup of the overall city, meaning that a grid of SABs will simply reproduce the existing segregation in the city. SABs which are 'gerrymandered' to be sprawling and irregular can draw students from across different ethnic neighborhoods in the city, resulting in a more diverse school. They use the 2009 SABINS database to show that this pattern largely holds, that irregularly drawn SABs almost always have diverse schools.

\end{itemize}
\section{Week 3}
\begin{itemize}

\item \cite{rey2011MeasuringSpatial}
\item This paper is concerned with advancing the study of spatial dynamics of neighborhoods, as opposed to neighborhood composition; i.e. spatial boundaries of a neighborhood vs. the people that live within it. The authors argue that while both are critical for understanding neighborhood change, the study of spatial boundaries is vastly underinterrogated compared to neighborhood composition. The framework presented ivolves regionalizing census tracts (using a Max-P algorithm) in Metropolitan Statistical Areas for two time periods and then investigating 22 neighborhood characteristic variables from census. Results were examined in the regionalized study area (neighborhoods) and at the /a priori/ level (tracts). The findings suggest that higher density, smaller land area, and more centrally located (centrality) neighborhoods tend to experience the highest degree of change. 

\item \cite{fu2020PromotingGeography}
\item This article asks Q1: Which disciplines are most quantitively interact with geography for the purpose of advancing sustainability science; Q2: How to best promote geograpic sciences in transdisciplinary methodologies /public policly/, /urban planninging/. The author collect data on frequencies of 11 words that appear with "geography" + "sustainability" in titles, keywords, and abstracts in publications of the ISI Web of Science from 2010 to 2019. The resulting associations between geography, sustainability are orgazied along different dimmensions, such as research objects, policy areas, and modelling methodology. The author then theorizea and describes 5 distinct spheres of research that geographers could pursue to advance the use of geographical methods into sustainability studies: Geographical processes, Ecosystem services and human wellbeing, Human-Environmental Systems, Sustainable development, and Geo-data and models for sustainability.

\end{itemize}
\section{Week 4}
\subsection{Theme: Big Data}
\begin{itemize}

\item \cite{thatcher2016DataColonialism}
\item This article contributes to theoretical understanding of the role of big data in capitalist production. Through End User License Agreements and myrriad smart devices and sensors, technology firms collect and aggregate data in a way that resembles David Harvey's Accumulation by Dispossession. Individual data points are abstracted away from the "lifeworld" and algorithmicly processed to generalize and predict purchasing and consumption patterns. "Social norms, aesthetic pleasures, and perceived values encourage the use of an increasing array of technologies equipped with sensors that quantify and then communicate data about previously private times and places to third-party actors." This process is termed "data colonialism," to contrast with the framing of technological advancements as "digital frontierism." Further investigation in this area for me is the role of big data in the "filtering" of news and personal information feeds from the basis of the agggregated digital identities created by dispossed data.

\item \cite{crampton2015CollectIt} 
\item This paper provides a study of the complications of the techonological advancement of big data. It draws primarily from examples of consequences drawn from the US Intelligence Agencies (IC) and are primarily divided into two categories: (geo)privacy and algorithmic security. "Big data are a matter of technologic /practices, epistimologies, and ontologies/." The article provides a detailed summary of key facts about the IC revealed by the Snowden documents; personel, budgets, specific operation details, and entertains the legal arguments surrounding things like bulk surveillance and 'incidental collection'. The author argues that corporate activities actively extend the state by increasing its reliance on the private sector; government purchases data, underwrites research, funds operations that increase government's depth of view. The paper touches on uses of big data by the government for warfare, particularly with drones in the war on terrorism. The author provides avenues for future research: Better histories of development of geospatial IC, better accounting for IC operations, better encryption and legal protections (informed consent about government surveillance.
  
\end{itemize}
\section{Week 5}
\begin{itemize}
  
\item \cite{laniyonu2018CoffeeShops}
\item This article presents an empirical analysis of the post-industrial policing hypothesis by operationalizing gentrification and applying spatial Durban models in New York City between 2010 and 2014. Spatial Durbin models an outcome of interest as a function of endogenous interaction effects, direct effects, and exogenous interaction effects. Effects are modelled based on differing theories to explain the spatial variation of policing, including rational-beauracratic theory, and conflict theory; racial threat, economic threat. Initial analysis yields support for all of these theories - Effectively, the Durbin modelling reveals that gentrification in a given tract is very strongly associated in increases in policing in neighboring tracts, but negatively associated in itself. As well, these effects vary with the extent of gentrifiation that has occurred - the author distinguishes between tracts inelligable for gentrification, tracts elligable, and post-gentrified tracts. These findings support the notion that police are utilized to drive undesired persons from a given area to make it more ammenible to the in-moving richer, whiter population. The poorer, darker population is corralled and heavily policed in adjacent tracts. Police ramp up their postindustrial policing practices in areas known to be undergoing change (i.e. experienced influx of 'undesirable' people).

\item \cite{sampson1997NeighborhoodsViolent}
\item This article is a study of the correlates of violent crime across varried neighborhood contexts, using the Project on Human Development in Chicago Neighborhoods dataset; the census tracts of which were regionalized into neighborhood clusters. The basic hypothesis is that collective efficacy, and through it informal social control, can explain the variation of violent crime across neighborhoods. Things like duration of tenacy and homeownership matter more than economic stratification or demographic factors (race). The methodology involves a hierarchical model of variations within persons, variations within neghborhoods, and variations between neighborhoods to get at correlates of social cohesion. After operationalizing collective efficacy, the author finds that collective efficacy is negatively related to violence in neighborhoods using regression modelings. Several additional tests were run to determine the extent that previous homicides, concentrated disadvantage, immigrant concentration and resident stability factor into violent crime, and found that collective efficacy was by far the largest effect. The results imply that collective efficacy can be meausured reliably at neighborhood scales and is relevant to the story of violent crime in Chicago neighborhoods. One major flaw of the study is that the operationalization of collective efficacy is done by collecting survey responses rather than actually observed.

\end{itemize}  
\section{Week 6}
\begin{itemize}
\item \cite{herring2014NewLogics}
\item This paper demonstrates a good deal of variation in large homeless encampments, and attempts to develop a typology. The study is the first to comparitavely examine variegated homelessness within a single analytical framework, consisting of 12 encampments across 8 municipalities on the west coast. The result is a topology of 4 kinds of encampments: /co-optation/, /accomodation/, /contestation/, and /toleration/. The author utlizes interviews with municipal administrators, non-profit actors, and the homeless residents, as well as living in the communities from a period fo 2009 to 2011. The typology reveals that the type of encampment is largely ``co-structured by policies of the state and the adaptive strategies of homeless people and their allies in their particular urban context.'' This is clear by contrasting the aesthetics, purpose, and results of Portland's Dignity Villiage (accomodation) with the prison-like structure and restrictions of Ontario's THSA (co-optation). These two examples represent the 'legal' end of the typology. The illegal encampments range from serving an explicitly political goal of bringing awareness to homelessness (contested) to mirroring representations of condensed poverty, including the open-air drug markets (toleration). This paper should be considered in a continuum of post-industrial policing in service of capital. Underlying many of the findings in this paper are the facts that municipalities take these actions at the behest of landed residents and commercial interests.

\item \cite{certoma2020DigitalSocial}
\item This paper is a review of a new concept called Digital Social Innovation (DSI), which refers to innitiatives that attempt leverage digital technologies to co-create solutions to a wide variety of social needs. DSI is associated with the development of``smart cites.'' The paper argues that a applying a critical geography lens to these initiatives could yield important perspectives about the power relationships involved. The author thus elaborates on 4 avenues of research that critical geographers should pursue. The suggested agenda starts with investigating DSI as networks of networks and deconstructing mainstream narratives about smart hyperconnected city in relation to the reproduction of capitalism. Are cities laboratories for technocratic governmental solutions? Are they incubators of citizen critical engagement or do they aid in the production of state and market power? This article is a little obtuse without any prior knowledge of some of the works referrenced. Further investigation: \emph{Manifesto for Digital Social Innovation (Chic 2020), Ind.ie, Mastodon, Digital Space (concept)}.

\item \cite{andreasson2022ErodingPublic}
\item This article demonstrates a system dynamics model to model the influence of the American Legislative Exchange Council (ALEC, the conservative lobbying group) on American legislative process through their increased membership. The political science literature guiding the model suggests that a shift in governance to 'network governance' is causing a reduction of resources in public institutions via the creation of new institutions. New problems result in the creation of new institutions, which absorb resources away from the existing institutions. The model demonstrates 6 feedback loops that reinforce and increase ALEC membership and influence in the legislative process. The authors then employ scenario analysis, wherein several hypothetical situations (based on 'what if'-ing real events, i.e. corporate backlash after the Trayvon Martin killing was linked to ALEC) are evaluated against a baseline scenario. The explanations provided by the model make sense and are intuitive, but the data driving the model is mostly implied, and unclear - the specific driving points are unstated. ALEC is very secretive with its data and does not generally assist in this expose. I might consider more research on system dynamics methodology in the event I want to model political behavior.

\end{itemize}  
\section{Week 7}
\begin{itemize}
  
\item \cite{richardson2019DirtyData}
\item This paper expands upon the term 'dirty data' to reflect the nature of data production in policing - derrived from corrupt and unlawful practices, and presents three case studies of police departments that were simulatniously under a consent decree or some other federal investigation while developing predictive policing systems. The idea is that these departments are under investigation as a result of their corrupt practices (including dirty data creation and utilization) while using those practices to inform their predictive systems. I read the Chicago section, which describes the Strategic Subject List, which was both ineffective at its stated goal and entirley informed by the corrupt and biases practices that preceeded it. The paper ultimately argues that dirty data exists systemically in the criminal justice system. The paper concludes with a brief, but concise discussion of the role of police as servants of capital (my interpretation), the dynamics of gentrification and the threat/consequences of unabatted use of dirty data in policing systems.


\item \cite{batty1997VirtualGeographya}
\item This paper is a widely cited typology of new frontiers created by the intersection of computers and geography. The author delineates and defines 3 distinct aspects of the digital world that are relevant to geography: cspace: the space within computers, cyberspace: the use of computers to communicate, and cyberplace - the infrastructure of the digital world. The author presents these aspects as interconnected and constantly influencing each other, along with the traditional concept of space/place. In this context, it seems that I would be interested most in cyberplace and the implications of its growth on public systems - how better systems can help improve society but more critically safeguarding against the dangers big data imply: the destruction of any concept of privacy or control over ones own life. The paper is relatively dated given the advance of technology in the previous two decades, and there is sure to be more recent works that I should investigate for any concrete phenomenon which might overlap with my research interests.
\end{itemize}


\begin{itemize}
\item \cite {speer2016RightInfrastructure}
\item This paper describes and advances an argument for the right of the homeless to a 'right to the city', originally articulated by Lefebver in 1996 (?); effectively arguing for a more dignified life decoupled from capitalist commodification of housing and sanitation infrastructure. The article details the conditions endured by Fresno's homelessness community using interviews and visits to encampments over several months. There is a discussion of municipal policy to displace and dehumanize the homeless while refusing to consider ammeliorating the conditions on the ground, primarily waste and sanitation services. A lot of the paper has to do with accesss to bathrooms for bodily autonomy. Public defication, urination, bathing, love-making, is dehumanizing, but conditions are such that nobody wants to provide (pay for) a solution for preventing it, so instead it is mobilized to advance arguments for destoying encampments. This paper has heartbreaking accounts of destruction of makeshift homes and treasured posessions to advance the normative arguments presented.

  
\end{itemize}
\section{week8}
\begin{itemize}
\item \cite{dewitt2022TwistedFate}
\item This comment is a law review of the California Environmental Quality Act (CEQA), in which the history of the evoluation of CEQA is described - the motivation, formulation, passage, and eventual mobilization and weaponization of the legislation to block construction projects, particularly as it relates to housing. It also highlights 3 major ways that California has attempted to circumvent, weaken the blockade of CEQA, mostly through state legislation: SB540, AB70, and AB73. These leverage other mechanisms in the state, such as the Regional Housing Needs Assessment - if the municipality isn't within projections to meet the needs as defined by the RHNA, certain CEQA requirements, time constraints, etc. could be bypassed. Because these bills needed leglislative concesssions to pass, such as requirements that developers pay workers prevailing wages, or that they bills fail to pass at all, they are largely inneffective for increasing the State's housing stock. The author concludes that solutions from within the stateare not viable, and that California should consider adopting policy fixes from other states: MA, MN, and NY, in ways that streamline the regulatory steps involved in housing development. These mostly involve trying to protect developments from frivolous or otherwise counterproductive CEQA lawsuits: removing anonymity, creating municipality-specific CEQA processes, enacting housing-friendly zoning statutes, and others. Ultimatley these are all subject to California's Supreme Court and have their own barriers. CEQA remains a impenetrable barrier for increasing the supply of housing in the state, seemingly in service of landed Capital to pursue commercial developments and upscale housing above multi-family dwellings, apartment buildings, and low income housing.

\end{itemize}
\section{Development Arrested}
\begin{itemize}

\item \cite{woods2017DevelopmentArrested}
\item Development Arrested is a seminal work that combines several methodologies to describe the history of development in the region known as the Mississippi Delta. The work is largely concerned with the power structures involved in the creation, action, and dissolution of administrative bodies in the region; but Woods also draws upon the development of culture in the region, and as such this book is also an epistomology of the blues, broadly defined.

\item \cite{isenberg2004SymposiumWoods}

\item \cite{tiefenbacher2019DevelopmentArrested}
\end{itemize}

\bibliographystyle{plainnat}
\bibliography{701}
\end{document}
