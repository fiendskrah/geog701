''computationalism, when harnessed to the authority of the state,
reifies a faith in instrumental rationality in which reason is the
ultimate arbiter of both truth and power.''

\subsection{Ubiquity}
Big Data is a catch-all phrase that has colloquially come to mean the
use of massive datasets to guide our decisions and policies. I am
interested in aspects of Big Data's impact on a variety of topics,
some of which are more theoretical and less quantifiable than the
subfields listed earlier. Crampton (2015) defines Big Data as "a
matter of technologic practices, epistimologies, and ontologies. This
definition captures the essence of Big Data as a practice rather than
simply concrete pieces of data. An implication of Big Data that
interestes me is that it centers the uninhibited, wanton harvesting of
data - pieces of information about humans, essentializing and
generalizing them. This serves a variety of purposes, some of which
could potentially be beneficial, some of which are immediately
nefarious, but all of which represent a major breach of private life
that merits investigation.

Batty (1997)


\subsection{Marxist perspectives}
Burell and Fourcade (2021) consider the topic from an explicitly
marxist perspective. They propose an extension of the class divide:
The coding elite (which is a good name) as the upper class oppressing
and extracting the wealth from the cybertariat (which is a bad name),
who comprises a whole swath of digital laborers. Something that is
unexplored (or unstated) about this formulation is the processes
through which these classes are produced. The authors begin with the
notion of almighty code - if you can touch, edit, or read the codes
that drive Big Data innovations, you inherently have more potential
than those who cannot. This is mostly correct, but increasingly, the
cybertariat class can and does touch code, and yet are relegated to
cybertarian jobs. I would argue that computer science know-how does
not necessairly elevate a person on their own merits. Inter and
intra-industry social capital still plays an outsized role in
determining if someone will advance in class status. This is a
potential avenue for further research.  Burell and Fourcade's article
is central to me on the subject due to its thorough review of
literature. They briefly describe the development of Sillicon Valley
and then investigate several issues brought about by the proliferation
of Big Data. For my purposes, I try to organize these into 1) Use of
algorithms in civic strucures and 2) use of algorithms in human
relations.

\subsection{Intersection with urban systems}
\subsubsection{Public policy digital government}
\begin{enumerate}
\item Police
  
Policing is a sort of venn diagram between several of my study
areas. I'm interested in the role of police from both a public policy
perspective and a human geography perspective. The police represent a
massive expendature by local municipalities, in stark contrast to
other services. Police are rightfully criticized for corruption,
excessive force, cruelty, racist practices, and the rest. And yet,
veneration of police and police culture is not uncommon.

Broadly, I want to include police abuses as a part of my research. A
clear intersection exists with enforcement of municipal policies
against the unhoused. There is a basic economic/public policy argument
for diverting funds away from police departments and towards
place-specific/regional specific policies to combat homelessness would
yield better results than the proliferation of police militarization
and continued instances of violence, corruption, and abuses that it
entails.

Local police and sherriff are only one factor in the perpetuation of
police violence; underappreciated are the roles of the courts, the
public officials that represent police interests in local politics, as
well as local, state, and federal legislation and executive policies
that all determine the logics of these sytems.

\begin{enumerate}
\item Local

Laniyonu (2018) performs an example of these mechanisms through the
framework of post-industrial policing, operationalizing gentrification
in New York City. I include this paper for the purposes of
highlighting its methodology. They use Spatial Durbin modeling to show
that gentrification can be predicted at a tract level by an increase
in policing in surrounding tracts, but a decrease within the tract
itself. As well, it should be considered part of the literature on use
of force by police; although the macro resolution of policing
abstracts the individual actions occuring to push the poor out of
gentrifying spaces, it should not be forgotten that these are often
violent encounters resulting in bodily harm, trama, or death.


Richardson (2019) is another important paper because it's a potential
avenue between two of my areas: policing and Big Data. Richardson
expands upon the term 'dirty data' to reflect the nature of data
production in policing - derrived from corrupt and unlawful
practices. Richardson analyzes a number of police jurisdictions that
develop predictive policing systems (such as LAPD's Compstat) WHILE
they are under a consent decree or under investigation by federal
authorities for civil rights violations. The paper details three such
cases, wherein the policing systems developed, whose ostentiable
purpose is to abate bias in the police department, are on their face
corrupt or biased, while also being ineffective at it's goal (usually
a tool for cops to replace the percieved mechanism of
corruption/abuse).

\item National

NSA (Snowden)

Crampton (2015) writes about complications related to Big Data using
the United States intelligence community (IC) as a case study.
  

\end{enumerate}
\end{enumerate}

\begin{enumerate}
\item Not police
  
\begin{enumerate}
\item Local

(some) Municipalities recognize that Big Data has immense potential to
  improve the quality and timelieness of services, and attempt to
  adopt technological solutions/improvements to previously analog
  processes.

Certoma (2020) elaborates a research agenda for digital social
innovations (DSI), which refer to initiatives that leverage digital
technologies to co-create solutions for a wide range of social
needs. They cite a few examples from the European context: The reusing
of abandoned buildings, and the organization of new commons. Of
particular interest to my research is the use of DSI in creating local
sharing economies in neighborhoods. In theory, this constitutes a
major element of a digital neighborhood dimension.

\item National

While I can assume that the federal government pursues Big Data
solutions for much the same reasons that municipalities do, I have not
yet read enough to provide any specific examples.
  
\end{enumerate}
\end{enumerate}

\subsection{Private enterprise}
\subsubsection{Advertising}

Williams (1961)
