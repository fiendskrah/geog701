\subsection{Galster's neighborhoods}

`Neighborhoods' are a critical concept in urban social science,
underlying investigations of how spatial contexts affect behavior and
outcomes as well as how space changes over time. Yet, a concrete
quantification of what constitutes a neighborhood is not universally
agreed upon. \cite{galster2001NatureNeighbourhood} attempts to define
`neighborhood' in a way that lends itself to quantification:
``Neighbourhood is the bundle of spatially based attributes associated
with clusters of residences, sometimes in conjunction with other land
uses.'' These attributes (or dimensions) include characteristics of
the natural and human-constructed environment, but also demographics,
socio-interactive characteristics, and sentimental value. Galster's
goal in this articulation is enabling researchers to test hypotheses
and construct predictive models of neighborhood change.

\subsection{Neigborhoods are Dynamic}

Galster's work helps us to delineate neighborhoods as they stem from
perceptions of their residents or governments, but they do not
explicitly understand temporal dynamics of neighborhoods. In other
words, it begins to address the number of issues arising from taking
administrative boundaries such as census tracts as \emph{a priori}
neighborhoods, but does not appreciate that the spatial structure of
neighborhoods - in addition to its physical and demographic
characteristics - are dynamic and can change over space and
time. \cite{rey2011MeasuringSpatial} argue that both of these
dimensions are critical for understanding neighborhood change, are are
likely to differ depending on the topic of neighborhood effects being
investigated. As well, they demonstrate that the study of spatial
boundaries are underappreciated in the scope of the larger literature.

\subsection{Externality Spaces}

Resident perceptions are a key factor in neighborhood quantification.
\cite{galster1986WhatNeighbourhood} describes 'externality spaces,'
which are the \emph{quantified} individual perceptions of the
previously described neighborhood dimensions. Specifically, a person's
externality space is the ``area over which changes in one or more
[dimensions] initiated by others are perceived as altering the
well-being [or use-value]'' of the
person. \cite{galster2001NatureNeighbourhood} Quantification of
neighborhood spaces by individual perspectives means that two people
living next door to each other could have differing views about where
their neighborhood begins and ends. As well, we might consider
administrative boundaries drawn by government agencies, such as census
tracts or school districts, as additional perspectives - the
difference being that administrative boundaries are often tied
directly to public policies, making them useful for policy analysis.

Galster also articulates three features of externality spaces against
which differing quantifications can be measured: \emph{congruence}, or
the degree to which an externality space corresponds to a
predetermined geographical boundary; \emph{generality}, or the degree
to which different neighborhood dimensions correspond; and
\emph{accordance}, or the degree to which externality spaces for
different individuals correspond.

\subsection{Towards a Dissertation}

Quantifying dynamic neighborhood dimensions through various
externality spaces approaches meaningful delineations of
`neighborhood.' The application of Galster's formulations to public
policy analyses yields methodologies to granularly measure policy
impacts across time and space. Therefore, this paper will explore
intersections in the neighborhood literature with public policy
analysis in an attempt to find gaps and articulate a research agenda.

First, we investigate applications to school systems. The neighborhood
effects on educational outcomes and vice versa -- the
school-neighborhood nexus -- is the subject of an extensive body of
literature, yet the spatial structure of the school-neighborhood nexus
and its potential to inform policy remains under investigated.

Next, we explore applications of Galster's formulation of
`neighborhood' to homelessness. Because Galster relies heavily on
quantification via market valuation of residences, quantifying the
`homeless dimension' of neighborhoods raises several complications.

Finally, we attempt to draw together a number of policy issues through
the theoretical lens of `computationalism,' or the belief that
computer processes \emph{can and must} underwrite social organization
and resource allocation.  To the computationalist, the inexorable
advancement of technology has imbued the computer (or more
specifically, the algorithm) with almost supernatural properties and
the capacity to solve all of the world's problems. Here, we explore
implications on social relations and civic structures by the
increasingly algorithmic logic that drives public policy.
