\subsection{Discuss the definition}
'Neighborhoods' are a critical concept in urban social science,
underlying investigations of how spatial contexts affect behavior and
outcomes as well as how space changes over time. Yet, a concrete
quantification of what constitutes a neighborhood is not universally
agreed upon. George Galster (2001) attempts to define 'neighborhood'
in a way that lends itself to quantification, including several
dimensions of neighborhoods which are spatially contained, and against
which hypothesis can be tested: "Neighbourhood is the bundle of
spatially based attributes associated with clusters of residences,
sometimes in conjunction with other land uses”. These attributes
(dimensions) include characteristics regarding the natural and
human-constructed environment, but also demographics,
socio-interactive characteristics, and sentimental value. While
Galster's dimensions are not the end of neighborhood quanitification,
they are a suitable starting point for this literature review. In
particular, this review will explore dimensions that intersect with
specific areas of public policy: A) school systems, which has a very
straightforward application to galster's neighborhoods through the use
of discrete geographic units; 2) homelessness, which complicates
quantification through Galster's neighborhoods due to it's reliance on
residencies; and 3) policing, as a specific subject to study variation
across neighborhood units.

\subsection{Externality spaces}
Galster describes three kinds of externality spaces

\subsection{Other approaches to defining neighborhoods}

