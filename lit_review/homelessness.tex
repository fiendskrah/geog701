Galster's emphasis on residence valuation as a focal point for
neighborhood quantification is complicated by the continued
proliferation of homelessness.  In this review, we explore potential
avenues from which to begin quantification. The unhoused are a
notoriously difficult population to study in any discipline and vary
widely from geography to geography. In particular, we seek to quantify
where and how the homeless work to carve out a daily existence, and
the effects of policies that both ameliorate and worsen homelessness.

The persistence of homelessness presents a quandary for neighborhood
quantification and effects modeling by challenging the definition of
`residency', which generally is a house or a group of homes, each of
which has different measurable attributes that determine their
perceived value, which spills over and combines with other units into
valuation of the larger area. Because the nature of residencies in
homelessness can be tenuous, quantification of a neighborhood
homelessness ``dimension'' \citep{galster2001NatureNeighbourhood} is
best suited to focus on the more permanent structures, or encampments,
of homeless communities as opposed to individual homeless
persons. This is not to downplay the individual homeless as less
important (and indeed these are the members of our communities who
most badly need help), but quantitative neighborhood modeling is
simply the wrong tool to address this problem. Qualitative methods in
other disciplines, specifically policy analysis and sociology, are
much better suited to make inroads in the homelessness
epidemic.

However, given that people both produce and consume their
neighborhoods \citep{galster2001NatureNeighbourhood},the individual
homeless might be better understood as \emph{attributes} of other
neighborhood dimensions, such as affecting the perceptions of
homeowners by `lowering' the value of the area, in a quantitative
spatial framework. Neighborhood encampments on the other hand, as we
will discuss below, represent a decision by policymakers to guide (or
push) the homeless to specified areas of a city, usually away from
areas that are targetted for upscale development. As such, encampments
have discrete borders where various `effects' can be granularly
quantified and tested against hypotheses.

Ultimately, quantification of the homelessness epidemic is a pedantic
exercise if it is not concerned with furthering research on the
ending homelessness through community actions and government
policies. \cite{belcher1991ThreeStages} provide a distinction between
different stages of homelessness which is widely accepted by social
workers and others who seek to end homelessness: marginal (or
episodic) homelessness, recent homelessness, and chronic
homelessness. These delineations are useful for crafting policies
aimed at addressing different stages, as some interventions are more
effective than others depending on which stage the person is in. The
remainder of this section will mostly be concerned with the
chronically homeless, as these are the people who are typically found
within encampments and are the most difficult to rehabilitate into
society.

In respect to different types of interventions, it can be assumed that
for the chronically homeless, more generous, longer-term assistance
policies are more effective at reducing homelessness than transitional
housing or short-term assistance. \cite{gubits2018WhatInterventionsa}
show that long-term housing vouchers are far more effective in
preventing a regression into homelessness than short-term housing
vouchers or transitional housing. This is a robust study that measures
impacts across dimensions of housing stability, family stability,
well-being of homeless parents and of their children, and
self-sufficiency. This study was performed on over 2000 participants
representing regions all across the United States, and employing pairwise
comparisons when possible.

\subsection{A Typology of Homeless Camps}

\cite{herring2014NewLogics} posits a 4-pronged typology of
homeless camps on the west coast of the United States. Herring divides
camps along dimensions of legality and control. The illegal dimensions
consist of contested camps (such as protest-guided tent cities) and
tolerated camps (those that are not legally sanctioned but tolerated
by authorities for pragmatic reasons); while legal camps can be either
accommodated (legally sanctioned camps that attempt to provide a link
between the unhoused and the potential of getting off the streets) or
co-opted (camps that have been effectively taken over by the
municipality, usually mirroring the conditionality of service
associated with homeless shelters and all the problems therein).

This typology contains a great deal of heterogeneity between different
camps that fall within the same categorization. As well, underexplored
is the idea that the differing camps may constitute a continuum rather
than just a typology. For a hypothetical example, the brutality
involved in the clearing of contested camps could cause enough unease
in the general population that they choose to pressure municipalities,
who change course and tolerate the camp. Then, the
tolerated camp allows social workers to experiment with different
styles of aid, leading the municipality adopt a more
accommodating stance, assisting the campers in establishing a
permanent site, as such was the case with Portland's Dignity
Village. To stay with our hypothetical, the accommodated camp may lead
the general population to become frustrated with the extent of
generosity given to the unhoused through a discourse of deservedness,
leading them to pressure the city, who then take a more authoritarian
approach, co-opting ownership and management of the camp. A specific
avenue of research in the neighborhood literature is the therefore the
quantification of \emph{flows} from one type of camp to another,
exploring and identifying examples of the scenario described here.

The variation in camps speaks to the individual spatial contexts of
the camps having effects on behaviors and social outcomes - concurrent
with Galster's conception. But Herring's typology is concerned with
only the visibly unhoused, and only in concentrations exceeding the
threshold required for collaboration such that a camp is
established. This typology does not consider the visible but isolated
homeless, in addition to the far more numerous 'invisible homeless' in
the episodic or recent stages. What this typology does well is
characterize the relationship and stance of the municipality to its
homeless population.

\cite{speer2016RightInfrastructure} articulates a 'right to the city,'
using a 'rights' framework similar to right to food, effectively
arguing for a more dignified life decoupled from capitalist
commodification of housing and amenities. They highlight sanitation
infrastructure and conditions of the unhoused in Fresno using
interviews and visits to encampments to illustrate the dehumanization
of the unhoused by municipal policymakers. Much of this article
discusses heartbreaking accounts of destruction of makeshift homes and
other attempts by the unhoused to find some comfort and bodily
autonomy.

The unhoused are forced to perform bodily functions in public,
including urination, defecation, eating, bathing, love-making, in ways
that are inherently dehumanizing. Nobody likes to witness these
things, nobody likes that the homeless are forced to do them, but
conditions are such that nobody wants to provide (pay for) a solution
to preventing it, so instead these facts are mobilized to advance arguments and
policies that force the unhoused out of public sight. This hostility
to the homeless is, at the time of this writing, inevitable from at
least some portion of the general population, which may or may not
influence municipal policy. What is important is how city
administrators react to hostility, what policies are enacted
downstream of that reaction, and what measures are taken to
counteract reactionary logics that lead to anti-homelessness policies.

Both Herring and Speer reference the use of force by municipalities
(police) to remove the unhoused from public spaces, whether an
established encampment or an individual using the restroom in a public
business.

\subsection{The Role of Economic Development in Municpal Homelessness Policy}

In many ways, the proliferation of homelessness depends largely on
scarcity of housing, specifically \emph{affordable} housing. Just as
neighborhood quantification generally is understood through housing
stock, the variations and impacts to the housing market have
reverberating effects across neighborhoods. As well, all actors in a
position to be concerned about neighborhood valuation are incentivized
to gamify and advance policies that increase valuation at any cost.

\cite{dewitt2022TwistedFate} demonstrates that the California
Environmental Quality act (CEQA), which requires the production of
enviornmental impact reports for new construction projects in the
state, has been effectively hijacked to delay and halt development on
undesirable (low-income) housing. Municipalities and residents don't
want to allow lower-income housing in their neighborhoods because they
perceive such construction as lowering the monetary valuation of their
neighborhood, a phenomenon known as NIMBYism (not in my
backyard). Therefore they utilize the power of CEQA in tandem with
other policies to stop development that could ameliorate homelessness
and lower housing prices more generally in favor of development that
will drive commercial traffic. A particularly egregious example is
presented in the case of SoFi stadium in Los Angeles, was was
fast-tracked to completion shortly after being introduced to the city
council, meanwhile significantly less costly affordable housing
projects face seemingly insurmountable blockages.

Development plans and specific projects represent decisions by
municipalities target discrete regions of their cities at the expense
of others. Concentrating affluence and commerce in one area
necessarily requires the removal of poorer humans and less profitable
buildings, a process known as gentrification. One theory about the
process of gentrification is that it happens on the receiving end of
of police mobilization - the post-industrial policing theory: the
police are utilized to drive undesirable (usually poorer, browner)
people out of areas that are targetted for
development. \cite{laniyonu2018CoffeeShops} presents an empirical
analysis of the post-industrial policing theory by modeling and
operationalizing gentrification. The author employs spatial Durban
models on New York City census tracts from 2010 to 2014 to show that
gentrification is strongly positively associated with increases in
policing in \emph{adjacent} tracts, rather than the gentrifying tacts
themselves, which have a negative association with police
presence. This supports the hypothesis that undesirable populations are
pushed to the periphery of gentrifying areas to make room for more
affluent populations. A relevant theory here is the so-called
`creative classes' theory, posited by Richard
Florida. \cite{florida2003CitiesCreative} presents a summary of this
theory, which argues that \emph{social capital}, or the communal ties
that bind us, is much less predictive of economic growth
than \emph{human capital}, or the skills and experience of individual
people. Florida's arguments have not aged gracefully - they require
that these rather ephemeral concepts are operationalized through
indexing census data, which is itself not without issues as described above, but are presented
in a rather opaque manner as well (one example is the use of a `gay
index' as an indicator of if a city has become more or less tolerant,
but the gay index is derived from self-reported census data
in \emph{the year 2000}. Florida describes homosexuality as ``the
final frontier of diversity.''). However, the downstream
post-industrial policing theory is much more robustly supported by
quantitative methods, which suggests that while the `creative classes'
theory may not be the definitive reasoning, the current stage of
capitalism has resulted in cities mobilizing police to arrange,
organize, and manage populations in a way that favors redevelopment for
the express purpose of extracting higher profits and rents.

In the following section, we further explore dynamics of policing and
other changes stemming from the inexorable march of technological
advancement.
