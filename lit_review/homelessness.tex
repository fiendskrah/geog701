The hostility of municipalities and taxpayers towards the homeless
through active and passive policy is violent and something I want to
include in my research. The homeless represent the extreme end of
poverty, the most deprived among us, the most in need of help, and
many people in the U.S. are only a few bad months away from being
there themselves.

\subsection{Emphasis on residential structures}
Galster's emphasis on residences as focal points for neighborhood
dimensions leaves the issue of homelessness, of particular concern for
Californians, as a major gap in neighborhood conception. Herring
(2014) posits a 4-pronged typology of homelessness camps on the west
coast of the U.S. that can prove useful for filling this gap. Herring
divides camps along dimensions of legality, with the illegal side
consisting of contested camps (such as protest-guided tent cities),
tolerated camps (those that are not legally sanctioned but tolerated
by authorities for pragmatic reasons); while legal camps can be either
accomodated (legally sanctioned camps that attempt to provide a link
between the unhoused and the potential of getting off the streets) or
co-opted (camps that have been effectively taken over by the
municipality, usually mirroring the conditionality of service
associated with homeless shelters and all the problems therein). This
typology contains a great deal of variety even between different camps
that fall within the same categorization, and may constitute a
continuum in itself - contested camps could cause enough pressure that
municipalities relent and accomoadate the campers with assistance in
establishing a permament site, as such was the case with Portland's
Dignity Village. The variation in camps speaks to the individual
spatial contexts of the camps having effects on behaviors and social
outcomes - concurrent with Galster's conception. But this typology is
concerned with only the visibly unhoused in concentrations passed the
threshold required for collaboration amongst the campers. A
homelessness dimension comporting with Galster's definition needs to
consider the visible but isolated homeless, as well as the far more
numerous 'invisible homeless' - those who are managing well enough to
couch surf or live in vehicles.

Speer (2016) Articulates a 'right to the city,' using a 'rights'
framework similar to right to food, effectively arguing for a more
dignified life decoupled from capitalist commodification of housing
and ammenities. They highlight sanitation infrastructure and
conditioins of the unhoused in Fresno using interviews and visits to
encampments to illustrate the dehumanization of the unhoused by
municipal policy makers. Much of this article discusses heartbreaking
accounts of destruction of makeshift homes and other attempts by the
unhoused to find some comfort and bodily autonomy. The unhoused are
forced to perform bodily functions in public, including urination,
defication, eating, bathing, love-making, in ways that are inherently
dehumanizing. Nobody likes to witness these things, nobody likes that
these people do them or are forced to do them, but conditions are such
that nobody wants to provide (pay for) a solution to preventing it, so
instead it is mobilized to advance arguments and policies that force
the unhoused out of public sight.

Both Herring and Speer reference the use of force by municiaplities
(police) to remove the unhoused from public spaces, whether an
established encampment or an individual using the restroom in a public
business.

\subsection{Housing market econometrics in municpal homelessness policy}
Dewitt (2022)

