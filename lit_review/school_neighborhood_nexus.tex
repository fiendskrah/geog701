\subsection{School Catchments}
My entry to this field began with considering the spatial elements of
schools. Of particular interest to me is the intersection - the nexus
- of schools and neighborhoods. Schools, under Galster's definition,
fall within the 'taxes and services' dimension of
neighborhoods. However, unlike neighborhoods, schools have discrete
boundaries that are drawn by a governing body that determine which
students will attend that school - called school attendance boundaries
or school catchment zones. As well, these boundries fall within the
geography of larger (also drawn) school districts, which themselves
fall within the boundaries of individual cities and counties. These
nesting, discrete boundaries have implications for how we concieve of
different spaces, and the higher resolution spaces have stronger
impacts for how we concieve of individual neighborhoods; and
crucially, they lend themselves well to measurements and hypothesis
testing via spatial statistical methods.

School catchment zones are considered from a variety of perspectives
depending on the context. Saporito and Van Riper (2016) investigates
whether or not the particular regularity of the catchment zones across
the US has any implication for reducing or exasperating segregation,
similarly to how congressional districts are drawn to capture a
particular voting population, a practice known as gerrymandering. The
find that on average, irregularly drawn catchments tend to have lower
levels of segregation than regular (rectangular) catchments,
suggesting that policy makers draw irregular catchments to increase
diversity in their schools. However, this could be a function of
boundary drawing in different urban contexts - urban cores tend to be
smaller with more heterogenous populations, whereas rural areas
necessarily need to be larger and have a more homogenous population.

Monarrez and Chien (2021) wrote an extensive report for the Urban
Institute that explicitly analyzed school catchments as a mechanism
for segregation, and highlighted cases in the U.S. where intense
pockets of segregation persist, even though the general levels of
school segregation have decreased in recent decades. While this report
performs a very interesting analysis linking these pairs of highly
segregated schools to the inequalities created by the New Deal's Home
Owners' Loan Corporation redlining policies using historical maps, it
unfortunately utilizes a privately sourced dataset, rendering the
findings impervious to reproduction.

A common theme across all of these studies is the potential for the
drawing of school catchments to function as an explicit tool for
policy makers to decrease segregation; 'good gerrymandering' (Owens
(various), Reardon (various)). I aim to contribute to this literature
through the employment of optimization modeling to draw school
catchment zones.

Despite their importance and relevance to the study of spatial
contexts and their effects, school catchments are very difficult to
study in the U.S due to a lack of institutional support for catchment
data - the latest nationally representative data year is the 2015-2016
school year, and even this is not without a variety of issues with
data accuracy which can cause issues when using software to perform
spatial statistical analyses on catchment data. Two well-published
scholars that are highly relevant to me for this subfield are Ann
Owens at the University of Southern California and Sean Reardon at Stanford University.

\subsection{Finnish context}
Kauppinen (2022) Bernelius (2019) study school catchment zones as
causal factors in intra-regional mobility and neighborhood segregation
in the context of Helsinki, Finland, which differs significantly in
political, economic, and cultural contexts. I cite them to contrast a
global perspective to my American-centric perspective. Notably, the
neighborhood contexts are different in that schools do not have
anywhere near the variation in quality that exists here. Though the
nordic countries are famed for their egalitarian civic structures,
they have their share of xenophobia regarding non-western
immigrants. Despite high quality schools uniformly across the region,
Bernelius 2019 is able to link urban mobility patterns and segregation
to school catchment zones through parent's perception of school
quality. Finish parents seek out higher quality schools (determined by
the number of native Finns, non-immigrants), up until they have
school-aged children, presumably prioritizing stability for the
student over their desires to find 'suitable' neighborhood
contexts. Similarly, Kauppinen (2022) finds that catchment boundaries
are a causal factor in intra-urban residential mobility using
regression discontinuity techniques. These findings are made possible
in part by the Finns maintaining datasets that are much higher quality
than exists in the U.S. and contain data about the entire population
as opposed to a sampling.
