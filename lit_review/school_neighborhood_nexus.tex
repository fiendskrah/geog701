In this section, we explore findings in the neighborhood effects
literature as it relates to school systems and discuss their
implications on public policy. Unlike vaguely defined neighborhoods,
education systems have discrete boundaries from which specific
policies are applied. The variation in policies, school
characteristics, and student outcomes between (and within) school
districts position school systems in a way that greatly benefits from
granular spatial and temporal policy analysis.

\subsection{School Catchments}

Individual schools have boundaries that determine which students will
attend that school, called school attendance boundaries or school
catchment zones. These boundaries nest within the geography of larger
school districts, which themselves fall within the boundaries of
individual cities and counties. These nesting, discrete boundaries
each have their own governing bodies which each apply their own
regulations and policies on top of existing state and federal
policies, all of which, in part, determine how we conceive of
different spaces. The higher-resolution (smaller) geographies have
stronger impact on how we conceive of individual neighborhoods, and
crucially, they lend themselves well to measurements and hypothesis
testing of policy impacts via spatial statistical methods.

Despite their importance and relevance to the study of spatial
contexts and their effects, school catchments are very difficult to
study in the U.S due to a lack of institutional support for catchment
data - the latest nationally representative data year is the 2015-2016
school year, and even this is not without issues. Beyond the
relatively `normal' issues that occur with spatial data such as opaque
and obscure coding logics, the School Attendance Boundary Survey
(SABS) data are effectively an amalgamation of data from disparate
educational agencies which may have different ways of gathering and
encoding this data, resulting in spatial-specific data issues such as
non-planar enforcement. These issues mean that the data require
additional processing before any spatial methodologies (such as
regionalization) can be applied. One major avenue of work that would
elevate the school-neighborhood nexus literature at large is
institutional support for consistent and consecutive SABS data-years
(akin to the American Community Survey).


\subsection{School Catchments and Segregation}

\cite{monarrez2021RaciallyUnequal} wrote an extensive report for the
Urban Institute that explicitly analyzed school catchments as a
mechanism for segregation, and highlighted cases in the U.S. where
intense pockets of segregation persist, even though the general levels
of school segregation have decreased in recent decades. While this
report performs a very compelling analysis linking pairs of highly
segregated schools to the inequalities created by the New Deal's Home
Owners' Loan Corporation redlining policies using historical maps, it
unfortunately utilizes a privately sourced dataset, rendering the
findings impervious to reproduction and validation.

In a similar vein,\cite{saporito2016IrregularlyShaped} investigates
whether or not the `regularity' of catchment zones across the US has
any implication for reducing or exasperating segregation, similar to
how congressional districts are drawn to capture a particular voting
population, a practice known as gerrymandering. The authors find that
on average, irregularly drawn catchments tend to have lower levels of
segregation than regular (compact) catchments.

While this might be a consequence of different urban contexts -
catchments in urban cores tend to be smaller with more heterogenous
populations, whereas catchments in rural areas necessarily need to be
larger and have a more homogenous population - it also suggests that
school administrators could be `gerrymandering for good'. Indeed,
advocacy for the use of boundary drawing as a tool for policymakers
to reduce segregation and increase diversity in schools is common
across this literature.

Two well-published scholars in this subfield are Ann Owens at the
University of Southern California, who studies trends in income and
racial segregation in U.S. schools and Sean F. Reardon, who heads the
Educational Opportunity Project at Stanford University, which is
interested in quantification of opportunity in different educational
regional contexts, primarily through variation in academic
performance. Together, \cite{owens2016IncomeSegregation} investigate
schools in the U.S. through these dimensions and conclude that income
segregation has increased in U.S. schools, measured as both
\emph{within and between} districts.


\subsection{Finnish Context}

\cite{kauppinen2022UnderstandingEffects} and
\cite{bernelius2019PupilsMove} study school catchment zones as causal
factors in intra-regional mobility and neighborhood segregation in the
context of Helsinki, Finland. They are presented here to contrast an
American-centric perspective by differing significantly in political,
economic, and cultural contexts. Notably, the neighborhood contexts
are different in that schools do not have anywhere near the variation
in quality that exists in the U.S. Though the Nordic countries are
famed for their egalitarian civic structures, they have their share of
xenophobia regarding non-western immigrants. Despite high quality
schools uniformly across the region, \cite{bernelius2019PupilsMove} is
able to model urban mobility patterns and segregation and finds school
catchment zones (via parent's perception of school quality) to be a
causal factor. Finish parents seek out higher quality schools
(determined by the number of native Finns, non-immigrants), up until
they have school-aged children, presumably prioritizing stability for
the student over their desires to find `suitable' neighborhood
contexts. Similarly, \cite{kauppinen2022UnderstandingEffects} finds
that catchment boundaries are a causal factor in intra-urban
residential mobility using regression discontinuity techniques. These
findings are made possible in part by the Finns maintaining datasets
that are much higher quality than exists in the U.S. and contain data
about the entire population as opposed to a sample.

\subsection{Spatial Congruence of the School-Neighborhood Nexus}

Forthcoming work by Rey and Knaap et al. (including the present
author) directly investigates the spatial structure of the
school-neighborhood nexus through Galster's \emph{congruence}. Instead
of taking census tracts as \emph{a priori} neighborhoods, The authors
regionalize census tracts using the \emph{max-p regions} algorithm to
create neighborhoods that represent geodemographic clustering in the
110 largest Core-Based-Statistical Areas in the United States. They
calculate congruence between these neighborhoods and school catchment
boundaries, and find correlates between congruence and spatial and
demographic characteristics of the spaces. These metrics are
constructed from both focal points: schools-to-neighborhoods and
neighborhoods-to-schools.

The results indicate that full congruence (parity between
neighborhoods and schools) is the exception, not the rule; congruence
correlates positively with size and circularity (regularity), and
negatively with density, though these relationships are by no means
linear. Another important finding is that as the proportion of the
Black population in a catchment grows, it tends to pull from
\emph{more} neighborhoods, while an increase in the White population
tends to pull from \emph{fewer} neighborhoods. As well, catchments
that pull from multiple neighborhoods tend to have higher levels of
diversity and are less regularly shaped, lending evidence to support
the claims of `good gerrymandering' suggested above.
